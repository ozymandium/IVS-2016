\documentclass[letterpaper, 10 pt, conference]{ieeeconf}  % Comment this line out if you need a4paper
%\documentclass[a4paper, 10pt, conference]{ieeeconf}      % Use this line for a4 paper
\IEEEoverridecommandlockouts                              % This command is only needed if 
                                                          % you want to use the \thanks command
\overrideIEEEmargins                                      % Needed to meet printer requirements.

\usepackage[pdftex]{graphicx}
% \usepackage
\usepackage[noadjust]{cite}

\title{\LARGE \bf
  Longitudinal Trajectory Planning and Tracking for Fixed-Path Autonomous Driving
}

\author{
  Robert G. Cofield $^{1}$,
  Rakesh Gupta $^{2}$,
  Ambarish Goswami $^{3}$,
  and
  David Bevly $^{4}$
  \thanks{
    * This work is supported by Honda Research Institute USA.
  }
  \thanks{
    $^{1}$ Robert G. Cofield is a graduate researcher at Auburn University's GPS \& Vehicle Dynamics Laboratory.
  }
  \thanks{
    $^{2}$ Rakesh Gupta ...
  }
  \thanks{
    $^{3}$ Ambarish Goswami ... 
  }
  \thanks{
    $^{4}$ David Bevly ...
  }
}


%%%%%%%%%%%%%%%%%%%%%%%%%%%%%%%%%%%%%%%%%%%%%%%%%%%%%%%%%%%%%%%%%%%%%%%%%%%%%%%%
%%%%%%%%%%%%%%%%%%%%%%%%%%%%%%%%%%%%%%%%%%%%%%%%%%%%%%%%%%%%%%%%%%%%%%%%%%%%%%%%
%%%%%%%%%%%%%%%%%%%%%%%%%%%%%%%%%%%%%%%%%%%%%%%%%%%%%%%%%%%%%%%%%%%%%%%%%%%%%%%%

\begin{document}

\maketitle
\thispagestyle{empty}
\pagestyle{empty}

%%%%%%%%%%%%%%%%%%%%%%%%%%%%%%%%%%%%%%%%%%%%%%%%%%%%%%%%%%%%%%%%%%%%%%%%%%%%%%%%
\begin{abstract}

When planning trajectories for an autonomous ground vehicle operating on roadways, it is common to employ the path-velocity decomposition (i.e., path planning is performed prior to planning the speed that the vehicle will take along the chosen path).
Given a desired path, we present a novel method of planning longitudinal trajectories (i.e., the time derivatives of position in the body-forward direction) along that path.
This method first segments the path using an arbitrary set of kinematic or dynamic constaints which may be functions of the desired path.
Each segement is then planned sequentially by choosing a piecewise constant jerk profile.
The jerk profile is intelligently chosen from a set of pre-solved profiles such that acceleration is continuous throughout the entire path.
The resultant trajectory plan can then be easily sampled to provide a reference for real-time vehicle control.
This method is shown to be efficient and reliable for use in online planning with a test vehicle.

\end{abstract}
%%%%%%%%%%%%%%%%%%%%%%%%%%%%%%%%%%%%%%%%%%%%%%%%%%%%%%%%%%%%%%%%%%%%%%%%%%%%%%%%

%%%%%%%%%%%%%%%%%%%%%%%%%%%%%%%%%%%%%%%%%%%%%%%%%%%%%%%%%%%%%%%%%%%%%%%%%%%%%%%%
\section{Introduction}
\label{sec:introduction}

... autonomous vehicles ... \emph{The 1st sentence is the hardest}.
One such task is computing a tentative reference trajectory plan for travelling from one point to another.
While the set of possible paths between any two points is infinite, it is typically heavily restricted by consideration of constraints such as traversability and legally available travel lanes, among others.
The intended speed, acceleration, and higher derivates of position with respect to time are also subject to constraints.
These may include legal speed limits, engine dynamics, desired sideslip limits, traction availability, and rollover concerns.

% intersection of path-planning constraints and speed-planning constraints is vehicle dynamics

% review of literature for a few paragraphs]
%  - searching planners (OMPL stuff)
%  - robot joint planning --> constant jerk intervals

% We will use the path-velocity decomposition
% mention using curvilinear CS
% 

Paragraphs here

Paragraphs here

% tracking
Once a trajectory plan is formulated, it must be utilized online in order for a reference signal to be sent to the controllers which govern steering, braking, and engine speed.
This is the trajectory tracking problem, which is discussed in Section \ref{sec:trajectorytracking}.

% mention using same cuvilinear cs scheme to project onto path

\section{Trajectory Planning}
\label{sec:trajectoryplanning}

For the purposes of this work, the objective of trajectory planning is to take a 2 dimensional (horizontal plane) path, and compute a series of constant jerk intervals for motion along the length of the path which is time optimal yet still falls within constraints on speed, acceleration, and jerk.
Integrating piecewise constant jerk over time yields acceleration which is piecewise linear over time, speed which is piecewise quadratic over time, and position which is piecewise cubic over time.
It is assumed that the path planner operates independently from the trajectory planner and tracker.
Prescribed initial and end conditions for position, speed, and acceleration must also be honored. 
This means that the desired trajectory plan is one dimensional, so using curvilinear coordinates will significantly simplify the solution.

% Need some lead in here?
% \subsection{Simple Case}

\begin{figure}[thpb]
  \centering
  \includegraphics[width=1.0\columnwidth]{graphics/7phaseprofile.png}
  \caption{7 phase constant jerk profile along curvilinear distance axis.}
  \label{fig:7phaseprofile}
\end{figure}

\subsection{Jerk Profiles}
\label{sec:jerkprofiles}

Consider, as a starting point, the case in which a vehicle is at rest and needs to accelerate to some travelling speed, maintain that speed for some distance, then brake to a stop at a prescribed end location.
Figure \ref{fig:7phaseprofile} shows this trajectory over time.
The problem is defined by the vector of known values $\bf{b}  = [v_i, a_i, L, v_m, v_f]$, being initial speed, initial acceleration, the total distance, speed limit, and the final speed, respectively.
If the magnitude of the desired peak acceleration and jerk ($a_m$ and $j_m$, respectively) are known, then the profile with $M$ jerk phases (7 in this case) can be generated by solving for the time intervals $[\Delta t_1, ..., \Delta t_M]$.
Jerk and peak acceleration may be different when reducing speed versus increasing speed to more closely mirror human driving behavior \emph{Citation}.
The remaining parameters yet to be set now become $\bf{a}_m = [a^+_m , a^-_m]$ and $\bf{j}_m = [j^+_m , j^-_m]$.
There are, however, multiple solutions for $\Delta t_2$, $\Delta t_4$, and $\Delta t_6$.
The proper solution must be chosen by eliminating negative and complex solutions.

It is obvious that there are many cases in which this solution will fail.
For instance, if $L$ is sufficiently short and $v_m$ is sufficiently high, then it will not be possible to ever reach $v_m$ with realistic values of $\bf{a}_m$ and $\bf{j}_m$.
The authors then propose that 4 other basic profiles be defined, and an algorithm constructed which judiciously chooses the appropriate profile for every segment defined by a vector $\bf{b}$.
The other profiles are explained below in section \ref{sec:jerkprofiles}.
A path is then a set of $N_s$ segments, each of which is defined by a set of parameters $\bf{b}_k, k = 1 ... N_s$ and can be solved indivually in sequence.
Determination of where to break a path into segments is discussed below in Section \ref{sec:pathsegmentation}.

There are a total of five potential jerk profiles which have been created to cover all feasible values in $\bf{b}$. All impossible scenarios are automatically ruled out.

\subsubsection{6 Phase}
\label{sec:6phase}

In the case mentioned above, the speed limit is too high for a solution to exist for a 7 phase profile.
The middle travel period is removed, so that the vehicle increases speed and then immediately decreases speed. 
The peak speed in this case has a solution, but cannot be manipulated directly.

\begin{figure}[thpb]
  \centering
  \includegraphics[width=0.7\columnwidth]{graphics/6phase_v(s).png}
  \caption{6 phase profile, displayed as speed vs. position along path}
  \label{fig:6phaseprofile}
\end{figure}

\subsubsection{4 Phase}
\label{sec:4phase}

When $v_i < v_m \simeq v_f$, the vehicle should increase speed, then maintain that speed for the distance remaining in the segment.

\begin{figure}[thpb]
  \centering
  \includegraphics[width=0.7\columnwidth]{graphics/4phase_v(s).png}
  \caption{4 phase profile, displayed as speed vs. position along path.}
  \label{fig:4phaseprofile}
\end{figure}

\subsubsection{Reversed 4 Phase}
\label{sec:reversed4phase}

When $v_i \simeq v_m > v_f$, it would be nonsensical to immediately brake.
The initial speed is maintained for a period before applying the brakes.

\begin{figure}[thpb]
  \centering
  \includegraphics[width=0.7\columnwidth]{graphics/4Rphase_all_derivatives.png}
  \caption{Reversed 4 phase profile, displayed as speed vs. position along path.}
  \label{fig:4rphaseprofile}
\end{figure}

\subsubsection{3 Phase}
\label{sec:3phase}

A change from one speed to another. 
This presents a difficulty in that either $L$ or $v_f$ may be enforced, but not both (there is no solution to enforce both).
As such, this profile is chosen for a segment in which it is impossible to completely arrive at the end speed given the initial speed. 

\begin{figure}[thpb]
  \centering
  \includegraphics[width=0.7\columnwidth]{graphics/3phase_v(s).png}
  \caption{3 phase profile, displayed as speed vs. position along path.}
  \label{fig:3phaseprofile}
\end{figure}

To decide which profile to employ, a cascade approach is used.
Analytic values of time intervals are evaluated for each profile by plugging in values of $\bf{b}$, $\bf{a}_m$, and $\bf{j}_m$ until all time intervals are positive and real. 
The order in which they are attempted is: 7 $\rightarrow$ 6 $\rightarrow$ 4 $\rightarrow$ 4R $\rightarrow$ 3.

\subsection{Path Segmentation}
\label{sec:pathsegmentation}

% make sure to incorporate extensibility to other constraints
The goal of the segmentation process is to divide a given path into multiple portions where each has a uniform upper limit on speed, acceleration, and braking.
There are 3 steps in the segmentation process:
\begin{enumerate}
  \item \emph{Ceiling calculation}, where $\bf{b}$, $\bf{a}_m$, and $\bf{j}_m$ are found for every point in the given path.
  \item \emph{Clustering}, in which path points are grouped sequentially.
  \item \emph{Consolidation}, in which a single $\bf{b}$, $\bf{a}_m$, and $\bf{j}_m$ is selected for each segment.
\end{enumerate}
After this is done, the process described above in Section \ref{sec:jerkprofiles} may be applied to each of the segments.

The ceiling calculation is a minimax problem.
Each of the 9 scalars is a minimum of several maxima computed from an arbitrary set of constraints.
For instance, ... \emph{put examples of things we didn't do...}
% How we actually did this.
For the purposes of developing a simple example application, two constraints will be used for $v_{m,k}$: legal speed limit ( $SL$, set constant for the whole path) and lateral acceleration limit ($a_{B,y}^{max}$. 
Lateral acceleration constraints are translated into longitudinal speed constraints using the approximation $v_{m,k}(a_{B,y}) = \sqrt{a_{B,y}^{max}/\kappa_k}$ for $k = 1, ..., N_p$.
For each path point, the value $v_{m,k}(a_{B,y}^{max})$ denotes the speed necessary to obtain the limit lateral acceleration, denoted by $a^{max}_{B,y}$.
The value $\kappa_k$ denotes the path curvature at each point.
For the sake of simplicity, the values of $a^+_{m,k}$, $a^-_{m,k}$, $j^+_{m,k}$, and $j^-_{m,k}$ are all set constant for the entire path, using values from \cite{Maurya2012,Hoberock1977,Long2000}.

This information is used to choose the segment boundaries in the clustering step.
Any number of clustering algorithms may be used here, and prudent choice of segment boundaries is arguably one of the most critical pieces of trajectory planning process.
For the present example, we will choose a simple univariate heuristic which essentially identifies curves in the roadway.
Each time that $v_{m,k} < SL$ becomes true, a curve is beginning, and a segment boundary is drawn.
Each time that $v_{m,k} = SL$ is once again true, the current curve has ended, and another segment boundary is drawn.
Figure~\ref{fig:course_highlight_turns} shows the result of this algorithm. 

\begin{figure}[thpb]
  \label{fig:course_highlight_turns}
  \centering
  \includegraphics[width=\columnwidth]{graphics/course_highlighted_turns.png}
  \caption{Clustering output for a short path when using an algorithm which draws segment boundaries around curves.}
\end{figure}

In the consolidation step, ...

\begin{figure}[thpb]
  \label{fig:consolidation_speed_ceiling}
  \centering
  \includegraphics[width=\columnwidth]{graphics/speed_ceiling.png}
\end{figure}

The power of this algorithm is that any set and number of constraints may be used ... \emph{etc., etc., ...}

\section{Trajectory Tracking}
\label{sec:trajectorytracking}

asdf

\section*{Acknowledgments}

Thanks.


%%%%%%%%%%%%%%%%%%%%%%%%%%%%%%%%%%%%%%%%%%%%%%%%%%%%%%%%%%%%%%%%%%%%%%%%%%%%%%%%

\bibliographystyle{IEEEtran}
\bibliography{HRI_internship_trajectory_planner_articles}

\end{document}