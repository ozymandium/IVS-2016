\documentclass[letterpaper, 10 pt, conference]{ieeeconf}  % Comment this line out if you need a4paper
%\documentclass[a4paper, 10pt, conference]{ieeeconf}      % Use this line for a4 paper
\IEEEoverridecommandlockouts                              % This command is only needed if 
                                                          % you want to use the \thanks command
\overrideIEEEmargins                                      % Needed to meet printer requirements.

\title{\LARGE \bf
  Longitudinal Trajectory Planning and Tracking for Fixed-Path Autonomous Driving
}

\author{
  Robert G. Cofield $^{1}$,
  Rakesh Gupta $^{2}$,
  Ambarish Goswami $^{3}$,
  and
  David Bevly $^{4}$
  \thanks{
    * This work is supported by Honda Research Institute USA.
  }
  \thanks{
    $^{1}$ Robert G. Cofield is a graduate researcher at Auburn University's GPS \& Vehicle Dynamics Laboratory.
  }
  \thanks{
    $^{2}$ Rakesh Gupta ...
  }
  \thanks{
    $^{3}$ Ambarish Goswami ... 
  }
  \thanks{
    $^{4}$ David Bevly ...
  }
}




%%%%%%%%%%%%%%%%%%%%%%%%%%%%%%%%%%%%%%%%%%%%%%%%%%%%%%%%%%%%%%%%%%%%%%%%%%%%%%%%
%%%%%%%%%%%%%%%%%%%%%%%%%%%%%%%%%%%%%%%%%%%%%%%%%%%%%%%%%%%%%%%%%%%%%%%%%%%%%%%%
%%%%%%%%%%%%%%%%%%%%%%%%%%%%%%%%%%%%%%%%%%%%%%%%%%%%%%%%%%%%%%%%%%%%%%%%%%%%%%%%

\begin{document}

\maketitle
\thispagestyle{empty}
\pagestyle{empty}

%%%%%%%%%%%%%%%%%%%%%%%%%%%%%%%%%%%%%%%%%%%%%%%%%%%%%%%%%%%%%%%%%%%%%%%%%%%%%%%%
\begin{abstract}

When planning trajectories for an autonomous ground vehicle operating on roadways, it is common to employ the path-velocity decomposition (i.e., path planning is performed prior to planning the speed that the vehicle will take along the chosen path).
Given a desired path, we present a novel method of planning longitudinal trajectories (i.e., the time derivatives of position in the body-forward direction) along that path.
This method first segments the path using an arbitrary set of kinematic or dynamic constaints which may be functions of the desired path.
Each segement is then planned sequentially by choosing a piecewise constant jerk profile.
The jerk profile is intelligently chosen from a set of pre-solved profiles such that acceleration is continuous throughout the entire path.
The resultant trajectory plan can then be easily sampled to provide a reference for real-time vehicle control.
This method is shown to be efficient and reliable for use in online planning with a test vehicle.

\end{abstract}
%%%%%%%%%%%%%%%%%%%%%%%%%%%%%%%%%%%%%%%%%%%%%%%%%%%%%%%%%%%%%%%%%%%%%%%%%%%%%%%%

%%%%%%%%%%%%%%%%%%%%%%%%%%%%%%%%%%%%%%%%%%%%%%%%%%%%%%%%%%%%%%%%%%%%%%%%%%%%%%%%
\section{Introduction}

... autonomous vehicles ... \emph{The 1st sentence is the hardest}.
One such task is computing a tentative reference trajectory plan for travelling from one point to another.
While the set of possible paths between any two points is infinite, it is typically heavily restricted by consideration of constraints such as traversability and legally available travel lanes, among others.
The intended speed, acceleration, and higher derivates of position with respect to time are also subject to constraints. These may include legal speed limits, engine dynamics, desired sideslip limits, tractions availability, and rollover concerns.

% intersection of path-planning constraints and speed-planning constraints is vehicle dynamics

% review of literature for a few paragraphs]
%  - searching planners (OMPL stuff)
%  - robot joint planning --> constant jerk intervals

% We will use the path-velocity decomposition
% mention using curvilinear CS
% 

% tracking
Once a trajectory plan is formulated, it must be utilized online in order for a reference signal to be sent to the controllers which govern steering, braking, and engine speed. This is the trajectory tracking problem.

% mention using same cuvilinear cs scheme to project onto path

\section{Trajectory Planning}

asdf

\section{Trajectory Tracking}

asdf

\section{Implementation}

% How much do we really want to put here?

% Show tracking performance? It's really more a function of the 

%%%%%%%%%%%%%%%%%%%%%%%%%%%%%%%%%%%%%%%%%%%%%%%%%%%%%%%%%%%%%%%%%%%%%%%%%%%%%%%%



\end{document}